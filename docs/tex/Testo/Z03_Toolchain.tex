\section{Tools}
\hypertarget{section::\theHsection}
Per realizzare il nostro progetto ci siamo avvalsi dei seguenti tool:
\begin{enumerate}
\item termux: terminale per mobile
\item python: linguaggio di programmazione usato per "runnare" il server virtuale
\item http-server: server virtuale per runnare siti web
\item Visual Studio: IDE di sviluppo general purpose
\item SublimeText: text editor per sviluppo general purpose
\item pkg: package manager per terminale mobile
\item pip: python package manager
\item Mozilla Firefox: Web Browser
\item Safari: Web Browser
\item Chromium : Web Browser
\item Konsole: terminale per ambiente GNU/Linux
\item tmux: multiplexer di finestre di terminale
\item gigle: software per GUI git, gestione branch e versioning
\item inspector: web browser debugging tool
\item evince: pdf document viewer
\item inkscape: editore di immagini e documenti raster
\item Git Cola: GUI software per la gestione dei branch e versioning git
\item Mocha: libreria per VueJS testing
\item Synaptic Package Manager: package manager per sistemi GNU/Linux
\item Jest: libreria per VueJs testing
\item ESlint: analisi statica del codice
\item Mega: servizio di archiviazione e backup online
\item Brackets: IDE per sviluppo HTML e CSS
\item WireShark: software per catturare ed analizzare pacchetti scambiati durante le interrogazioni alle API
\item Github: sito di hosting di repository
\item WebPack: module bundler e minimizer per applicazioni JavaScript
\item SourceTree: client di gestione versioning con GUI git
\item Eclipse: IDE per la programmazione in Java
\item WebStorm: IDE per programma in VueJS Framework di Javascript
\item NodeJS: gestore di moduli e librerie per sviluppo di siti web
\item Yarn: package manager
\item Bitbucket: servizio web per la gestione di repository git
\item Overleaf: servizio web per la creazione e archiviazione di documenti \LaTeX. Consente inoltre la collaborazione di più utenti sullo stesso documento in tempo reale
\item Draw.io: servizio Web per la creazione di grafici fra cui anche i diagrammi UML
\item Telegram: app di messaggistica
\end{enumerate}

\section{Librerie}
\hypertarget{section::\theHsection}
Per la realizzazione del nostro progetto ci siamo avvalsi delle seguenti librerie:

\begin{enumerate}
\item Leaflet
\item Vue2 leaflet
\item vue places
\item leaflet routing machine
\item leaflet control geocoder
\item axios
\end{enumerate}